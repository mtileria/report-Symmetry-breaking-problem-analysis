\section{Introduction}
\label{cap:1}

A symmetry in distributed systems occurs when all processes in a network are in the same state, possibly with an unique identifier. This state of symmetry should be broken before beginning any computation, this problem is known as Symmetry Breaking. Some examples of symmetry breaking problems are computing the maximal independent sets \textit{(MIS)} , maximal matching, vertex colouring, ruling sets and leader election. This project focuses on finding the maximal independent set on an undirected graph using the local model \cite{linial1992locality}.

In the local model, each vertex of the graph is occupied by a processor or process and has an unique identifier ID. The terms process and processor are used in the same context and are interchangeable in the literature. The processors only communicate with each other by sending messages, there is no notion of shared memory. In distributed networks locality means that in order to obtain the solution of a general problem, a processor only uses locally available data. In this model, computation is assumed to be reliable and synchronous. No faulty processors are permitted in this model and every processor executes the local algorithm in steps call rounds.  

An independent set \textit{IS} of an undirected graph is a subset S of nodes such that no two nodes in S are adjacent. An \textit{IS} is considered maximal if no extra node can be added to \textit{IS} without violating the independence. The \textit{MIS} problem is one of the most important problem in the area of local graph algorithm because many other problems like graph colouring can be reduced to it \cite{panconesi1992improved}. A maximum independent set \textit{MaxIS} is a \textit{MIS} of maximum cardinality, this problem, in contrast to \textit{MIS} is NP-hard. In this project a random-priority parallel algorithm developed by Yves \textit{et al.} \cite{yves2009optimal} is used in order to find the MIS of a given graph.


There are many applications of the \textit{MIS}, for instance: resource scheduling, topology control in wireless sensor networks \cite{basagni2001finding}, analysis of biological systems \cite{afek2013beeping}. Karp and Wigderson \cite{karp1986constructing} presented reductions from \newline others graph theory problems (Maximal Set Packing, Maximal Matching, 2-Satisfiability) to the \textit{MIS} and proved that the \textit{MIS} problem is in NC.


In practice, the majority of the distributed systems are asynchronous. However, protocols are easy to design and implement in the synchronous model. Once an algorithm has been designed in the synchronous model it can be transformed to an asynchronous algorithm. The main reason why synchronous algorithms are easy to design is that the model makes strong assumptions and put constraints that restrict the behaviour of the system.


There are many techniques to simulate a synchronous system over an asynchronous system. A synchronizer is a simulation from the synchronous system to the asynchronous system. These simulations can be local or global. In local simulations, processes only keep the synchronization with their neighbours. On the contrary, on global simulations all processes are synchronised by one master or root process.  For this reason, in global simulation all processes are always in the same round but in local simulation processes might be in different rounds, especially in large networks \cite{attiya2004distributed}. For this project, two well known synchronizers proposed by Awerbuch \cite{awerbuch1985complexity} are used: Alpha and Beta Synchronizer. Besides, a global synchronizer with a master processes that controls the synchronization was implemented to analyse the overhead generated by different techniques in terms of messages and time.




The rest of the report is organised as follow: The next section presents relevant work about symmetry breaking, maximal independent set problem and synchronizers. Section \ref{cap:2} presents the background of symmetry breaking and defines the MIS problem. Different approaches to simulate synchronous systems are discussed in Section \ref{chap:3}. A description of the simulator and a tutorial on how to use it is given in section \ref{chap:4}. The experimental settings of the simulations are explained in Section \ref{chap:5}. Section \ref{chap:6} gives the evaluation of results and conclusion. Finally, the professional issues are presented in Section \ref{chap:7}. 

\subsection{Background Research}
 
Deterministic and randomised algorithms are two well studied approaches to solve symmetry breaking problems. Usually, randomisation provides simpler and faster implementation than the deterministic counterpart. Deterministic algorithms always reach the solution of the problem, in contrast, randomised algorithms achieve termination with high probability. Relevant work, especially the ones related to \textit{MIS} problem are quoted below.

In \cite{luby1986simple}, Luby proposes the first solution to the \textit{MIS} problem using a simple randomised algorithm. The same year Alon \textit{et al.} present in \cite{alon1986fast} another randomised solution and made a comparison with others algorithms that were turned from randomised to deterministic. Both algorithms run in $O(\log |N|)$ in general graphs. In \cite{linial1992locality} Linial presents a $\Omega(\log* |N|)$ algorithm to find the MIS for n-cycle graphs. A deterministic solution was presented by Panaconesi \textit{et al.} in \cite{panconesi1996complexity} that run in $2^{O\sqrt{\log N}}$. More recently, many deterministic algorithms were presented by Barenboim and Elkin in \cite{barenboim2010sublogarithmic} with different running times for different network topologies. Schneider and Wattenhofer developed a $O(\log^* N)$ deterministic algorithm for graphs with bounded growth in \cite{barenboim2010sublogarithmic}. In \cite{yves2009optimal}, Ives \textit{et al.} present a variant of the original Luby's algorithm with optimal bit complexity for the messages sent over the network. This algorithm is used in this project for testing.

 Barenboim \textit{et al.} made in \cite{barenboim2016locality} a survey of symmetry breaking algorithms describing the main results achieved on the topic and proposed new algorithms for the most popular symmetry breaking problems: \textit{MIS}, maximal matching, vertex colouring and ruling sets. In \cite{johnson1985symmetry} Johnson and Schneider present a discussion to address the meaning of symmetry and introduce the concept of similarity.  
 

In \cite{awerbuch1985complexity}, Awerbuch described a general simulation technique to allow the implementation of algorithm in synchronous networks, referred as synchronizer. In the paper, the author presented three synchronizers: Alpha, Beta and Gama, which are a generalisation of the work of Gallager and Robert in \cite{gallager1982distributed}. Logical buffering is another technique to simulate synchronization and was independently developed by Welch in \cite{welch1987simulating} and Neier and Toueg in \cite{neiger1993simulating}. Others techniques were developed improving the synchronizer mention before under some restrictions, for instance, Peleg in \cite{peleg1989optimal} worked on multidimensional mesh network with two nodes in each dimension(hyper-cube).
