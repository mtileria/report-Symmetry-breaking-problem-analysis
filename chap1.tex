 One of the central problem in distributed computing is called Symmetry Breaking. This problem occurs when all processors in a network are in the same state, possibly with an ID to be uniquely identified. This state of symmetry should be broken before beginning any computation. Some examples of symmetry breaking problems are computing the maximal independent sets \textit{(MIS)} , maximal matching, vertex colouring, ruling sets and leader election. This project studies the problem of finding the maximal independent set on an undirected graph using the local model \cite{linial1992locality}.

In the local model, each node of the graph is occupied by a processor and has a unique identifier ID. The processors only communicate with each other by sending messages, there is no notion of shared memory. Locality in distributed networks means that in order to obtain the solution of a general problem, a process only uses locally available data. Computation is assumed to be reliable and synchronous. No faulty processors are permitted in this model and every processor executes the local algorithm in steps call rounds.

An independent set \textit{IS} of an undirected graph is a subset S of nodes such that no two nodes in S are adjacent. An \textit{IS} is considered maximal if no extra node can be added to S without violating the independent set. The \textit{MIS} problem is one of the most important problem in the area of local graph algorithm and many other problems like graph colouring can be reduced to it \cite{panconesi1992improved}. A maximum independent set \textit{MaxIS} is one of maximum cardinality, this problem, in contrast to \textit{MIS} in NP-hard. Examples of applications of the MIS of a graph are resource scheduling, topology control in wireless sensor networks \cite{basagni2001finding}, analysis of biological systems \cite{afek2013beeping}. In this project a random-priority parallel algorithm developed by Yves \textit{et al.} \cite{yves2009optimal} is used in order to find the MIS of a given graph and run the simulations.


In practice, most of the distributed systems are asynchronous, however, protocols are easy to design and implement in the synchronous model because the behaviour is more restricted. In this model, processors execute in lock steps as follow: Every processor partitions the algorithm execution in logical timing call round. In these rounds, a process can send messages to each neighbour, receive messages from its neighbours and perform a local computation. The main problem with asynchronous systems is that message delay is unbounded, there is no limit on how long a process should wait to determine that it received all messages from its neighbours. Once an algorithm is designed for the synchronous model, it can be simulated in a more realistic model like asynchronous system. 



There are many techniques to simulate synchronous system over an asynchronous message passing system. A synchronizer is a simulation from the synchronous system to the asynchronous system. Simulations can be local or global. In local simulations, processors only communicate with its neighbours, in consequence it is possible, especially in large networks, that some processors are in different rounds, this is not the case of global simulation in which there is one processor that controls the start of a new round and all processors are always in the same round. For this project, two well know synchronizers proposed by Awerbuch \cite{awerbuch1985complexity} are used. Besides, a global synchronizer was implemented to analyse the overhead generated by different techniques in terms of messages and time. 


The metrics used to measure the complexity of a distributed algorithm are the time and number of messages. The most simple way to measure time in the synchronous model is to count the number of rounds until the algorithm terminates. The message complexity consists in count the total number of messages sent by the processes. 

The rest of the report is organised as follow: Section 2 presents a list relevant related work about symmetry breaking, maximal independent set and synchronizers. Section 3 presents the background for the symmetry breaking and MIS problem. Different approaches to simulate synchronous distributed messages passing systems are discussed in Section 4. The experimental results and a brief tutorial on how to use the simulator are presented in Section 4. Section 5 gives the conclusion of the project and finally, the professional issues are presented in Section 6. 

\subsection{Background Research}
 
Deterministic and randomised algorithms are two well studied approaches to solve symmetry breaking problem. Usually, randomisation provide simpler and faster implementation than the deterministic counterpart. Deterministic algorithm always reaches the solution of the problem, in contrast, randomised algorithms achieve termination with high probability. Relevant work, especially the ones related to \textit{MIS} problem are quoted below.

In \cite{luby1986simple}, Luby proposes the first solution to the \textit{MIS} problem using a simple randomised algorithm. The same year Alon \textit{et al.} present in \cite{alon1986fast} another randomised solution and made a comparison with others algorithms that were turned from randomised to deterministic. Both algorithms run in $O(\log |N|)$ in general graphs. In \cite{linial1992locality} Linial present a $\Omega(\log* |N|)$ algorithm to find the MIS for n-cycle graphs. A deterministic solutions was presented by Panaconesi \textit{et al.} in \cite{panconesi1996complexity} that run in $2^{O\sqrt{\log N}}$, more recently, many deterministic algorithms were presented by Barenboim and Elkin in \cite{barenboim2010sublogarithmic} with different running times for different network topologies. Schneider and Wattenhofer developed a $O(\log^* N)$ deterministic algorithm for graphs with bounded growth in \cite{barenboim2010sublogarithmic}. In \cite{yves2009optimal}, Ives \textit{et al.} present a variant of the original Luby's algorithm with optimal bit complexity for the messages sent over the network, this algorithm is used in this project for testing purpose.

 

In \cite{awerbuch1985complexity}, Awerbuch present a general simulation technique to allow the implementation of algorithm in synchronous networks, referred to as synchronizer. In this paper, the author present three synchronizer Alpha, Beta and Gama, which are a generalisation of the work of Gallager and Robert in \cite{gallager1982distributed}. Logical buffering is another technique to simulate synchronization and was independently developed by Welch in \cite{welch1987simulating} and Neier and Toueg in \cite{neiger1993simulating}. Others techniques were developed improving the synchronizer mention before under some restrictions, for example, Peleg in \cite{peleg1989optimal} for multidimensional mesh network with two nodes in each dimension(hyper-cube).

 Barenboim \textit{et al.} made in \cite{barenboim2016locality} a survey of symmetry breaking algorithms describing the main results achieved on the topic and proposed new algorithms for the most popular symmetry breaking problems: \textit{MIS}, maximal matching, vertex colouring and ruling sets. In \cite{johnson1985symmetry} Johnson and Schneider present a discussion to address the meaning of symmetry and introduce the concept of similarity.  
\newline