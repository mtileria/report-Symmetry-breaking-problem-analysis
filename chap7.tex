\section{Self Assessment}
\label{chap:7}


In this section, I will present a personal evaluation of this final dissertation.  The assessment includes the progression of the project, acquired skills and a description of professional issues, in particular about correct citation and code acknowledgement.

% I started with this project at the end of May. I began to read about the symmetry breaking problem in distributed computing and implemented 2 leader election algorithms (studied in CS5860) for ring topologies. On the first week of June, I decided to work on the Maximal Independent Set problem. After read several papers and online materials, I started implementation of the algorithm proposed in \cite{yves2009optimal} with a global synchronisation technique. By the end of June, I began with the synchronizers Alpha and Beta. After I finished the synchronizers, I focused on the report and in minors changes to the implementation.

During this project, I was able to consolidate my knowledge of programming distributed systems. In the course CS5860, I had the opportunity to learn the Actor model in Elixir and implement programs on the distributed model assuming asynchronous communication. Besides, in the Wireless Sensor Networks course (CS5870), I implemented two gossip algorithm over a stack of Arduino and XBee. I learnt the importance of theoretical analysis for distributed algorithms. The proof of correctness and termination ensure that an algorithm is correct. In practice, debug an algorithm for large topologies is almost impossible, but one can rely on the proofs to make sure the correctness of the implementation. These experience helped me to understand the model of distributed programming and prepared me to face this dissertation. After completing this project, I feel more confident in my ability to implement and design distributed applications and feel that I gained a lot of practical experience and theoretical knowledge.     

% I learnt the importance of theoretical analysis for distributed algorithms. The proof of correctness and termination ensure that an algorithm is correct. In practice is also impossible  to debug an algorithm for big topologies but one can relay on the proofs to make sure the correctness of the implementation. 
% theoretical knowledge gain during the course help me to solve the problem. 

I used open-source code to generate the network topologies for the testbed. First, I used the open source library (GPLv3) \textbf{algs4} written in Java, which is part of the book \textbf{Algorithms $4^th$ Edition} of Sedgewick and Wayne. This library has a section for generating random graphs which include the model  Erd\~os--R\'enyi. Finally, I used another open-source implementation from a public GitHub repository under MIT license to produce the network topologies with the Stochastic Block model. It is essential to mention and acknowledge in the report any implementation that was not made by me. The correct citation was done in the section \ref{chap:5}. I had to spend some time reading about licensing to understand until what point I could reuse the software. The correct citation is also another important topic regarding ethical issues. I read the rules of the University on plagiarism and I used online tools to double check if there was any part of the report which was not properly cited. 

My contribution to the community is a new simulator of synchronous communication written in Elixir in which anyone can test synchronous algorithm without worrying about the underline asynchronous communication.  This simulator is in the public domain in GitHub \href{https://github.com/mtileria/SymmetryBreaking}. I decide to put this implementation under GPLv3 license, to make sure that this code will remain free to anyone.

% Plagiarism - correct citation, using code with acknowledgement.