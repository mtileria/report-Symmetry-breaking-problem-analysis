\documentclass[11pt]{article} % do not change this line
\input{BigDataStyle.txt}      % do not change this line
\usepackage{amsmath,amsfonts,amssymb,amsthm,latexsym,graphicx,caption,hyperref}
\usepackage[boxed,ruled,vlined,dotocloa,english,onelanguage]{algorithm2e}
\graphicspath{ {pictures/} }
\emergencystretch=5mm
\tolerance=400
\allowdisplaybreaks[4]

\theoremstyle{plain}
\newtheorem{theorem}{Theorem}[section]
\newtheorem{proposition}[theorem]{Proposition}
\newtheorem{corollary}[theorem]{Corollary}
\newtheorem{lemma}[theorem]{Lemma}
\newtheorem{problem}[theorem]{Problem}
\newtheorem{definition}{Definition}[section]

\theoremstyle{definition}
\newtheorem*{remark}{Remark}

\title{Analysis of Symmetry Breaking Algorithms in Distributed Synchronous Systems}
\author{Marcos Tileria}

\newcommand{\Programme}{MSc Distributed and Networked Systems}
% Computational Finance students: uncomment the next line
%\twodepartmentstrue

\begin{document}
\maketitle

\declaration

\begin{abstract}
  
 Symmetry Breaking is well studied problem in distributed computing. A distributed system present a symmetry when all processes share the same state. The goal of a symmetry breaking algorithm  is to break the state of equilibrium of a distributed system before the start of a computation. The objective of this dissertation is to analyse time and message complexity of distributed symmetry breaking algorithms on synchronous systems, in particular the Maximal Independent Set problem, which result to be one of the most important symmetry breaking problems. Three types of synchronization techniques are used to simulate a synchronous system over an asynchronous message passing system. A simulator develop in the Elixir was built in order to analyse the problem using different synchronizations techniques. The results of the simulation show that ..... and present the trade off among the synchronization techniques.... bla bla 
  

\end{abstract}



\section{Introduction}
\label{cap:1}

 One of the central problem in distributed computing is called Symmetry Breaking. This problem occurs when all processors in a network are in the same state, possibly with an ID to be uniquely identified. This state of symmetry should be broken before beginning any computation. Some examples of symmetry breaking problems are computing the maximal independent sets \textit{(MIS)} , maximal matching, vertex colouring, ruling sets and leader election. This project studies the problem of finding the maximal independent set on an undirected graph using the local model \cite{linial1992locality}.

In the local model, each node of the graph is occupied by a processor and has a unique identifier ID. The processors only communicate with each other by sending messages, there is no notion of shared memory. Locality in distributed networks means that in order to obtain the solution of a general problem, a process only uses locally available data. Computation is assumed to be reliable and synchronous. No faulty processors are permitted in this model and every processor executes the local algorithm in steps call rounds.

An independent set \textit{IS} of an undirected graph is a subset S of nodes such that no two nodes in S are adjacent. An \textit{IS} is considered maximal if no extra node can be added to S without violating the independent set. The \textit{MIS} problem is one of the most important problem in the area of local graph algorithm and many other problems like graph colouring can be reduced to it \cite{panconesi1992improved}. A maximum independent set \textit{MaxIS} is one of maximum cardinality, this problem, in contrast to \textit{MIS} in NP-hard. Examples of applications of the MIS of a graph are resource scheduling, topology control in wireless sensor networks \cite{basagni2001finding}, analysis of biological systems \cite{afek2013beeping}. In this project a random-priority parallel algorithm developed by Yves \textit{et al.} \cite{yves2009optimal} is used in order to find the MIS of a given graph and run the simulations.


In practice, most of the distributed systems are asynchronous, however, protocols are easy to design and implement in the synchronous model because the behaviour is more restricted. In this model, processors execute in lock steps as follow: Every processor partitions the algorithm execution in logical timing call round. In these rounds, a process can send messages to each neighbour, receive messages from its neighbours and perform a local computation. The main problem with asynchronous systems is that message delay is unbounded, there is no limit on how long a process should wait to determine that it received all messages from its neighbours. Once an algorithm is designed for the synchronous model, it can be simulated in a more realistic model like asynchronous system. 



There are many techniques to simulate synchronous system over an asynchronous message passing system. A synchronizer is a simulation from the synchronous system to the asynchronous system. Simulations can be local or global. In local simulations, processors only communicate with its neighbours, in consequence it is possible, especially in large networks, that some processors are in different rounds, this is not the case of global simulation in which there is one processor that controls the start of a new round and all processors are always in the same round. For this project, two well know synchronizers proposed by Awerbuch \cite{awerbuch1985complexity} are used. Besides, a global synchronizer was implemented to analyse the overhead generated by different techniques in terms of messages and time. 


The metrics used to measure the complexity of a distributed algorithm are the time and number of messages. The most simple way to measure time in the synchronous model is to count the number of rounds until the algorithm terminates. The message complexity consists in count the total number of messages sent by the processes. 

The rest of the report is organised as follow: Section 2 presents a list relevant related work about symmetry breaking, maximal independent set and synchronizers. Section 3 presents the background for the symmetry breaking and MIS problem. Different approaches to simulate synchronous distributed messages passing systems are discussed in Section 4. The experimental results and a brief tutorial on how to use the simulator are presented in Section 4. Section 5 gives the conclusion of the project and finally, the professional issues are presented in Section 6. 

\subsection{Background Research}
 
Deterministic and randomised algorithms are two well studied approaches to solve symmetry breaking problem. Usually, randomisation provide simpler and faster implementation than the deterministic counterpart. Deterministic algorithm always reaches the solution of the problem, in contrast, randomised algorithms achieve termination with high probability. Relevant work, especially the ones related to \textit{MIS} problem are quoted below.

In \cite{luby1986simple}, Luby proposes the first solution to the \textit{MIS} problem using a simple randomised algorithm. The same year Alon \textit{et al.} present in \cite{alon1986fast} another randomised solution and made a comparison with others algorithms that were turned from randomised to deterministic. Both algorithms run in $O(\log |N|)$ in general graphs. In \cite{linial1992locality} Linial present a $\Omega(\log* |N|)$ algorithm to find the MIS for n-cycle graphs. A deterministic solutions was presented by Panaconesi \textit{et al.} in \cite{panconesi1996complexity} that run in $2^{O\sqrt{\log N}}$, more recently, many deterministic algorithms were presented by Barenboim and Elkin in \cite{barenboim2010sublogarithmic} with different running times for different network topologies. Schneider and Wattenhofer developed a $O(\log^* N)$ deterministic algorithm for graphs with bounded growth in \cite{barenboim2010sublogarithmic}. In \cite{yves2009optimal}, Ives \textit{et al.} present a variant of the original Luby's algorithm with optimal bit complexity for the messages sent over the network, this algorithm is used in this project for testing purpose.

 

In \cite{awerbuch1985complexity}, Awerbuch present a general simulation technique to allow the implementation of algorithm in synchronous networks, referred to as synchronizer. In this paper, the author present three synchronizer Alpha, Beta and Gama, which are a generalisation of the work of Gallager and Robert in \cite{gallager1982distributed}. Logical buffering is another technique to simulate synchronization and was independently developed by Welch in \cite{welch1987simulating} and Neier and Toueg in \cite{neiger1993simulating}. Others techniques were developed improving the synchronizer mention before under some restrictions, for example, Peleg in \cite{peleg1989optimal} for multidimensional mesh network with two nodes in each dimension(hyper-cube).

 Barenboim \textit{et al.} made in \cite{barenboim2016locality} a survey of symmetry breaking algorithms describing the main results achieved on the topic and proposed new algorithms for the most popular symmetry breaking problems: \textit{MIS}, maximal matching, vertex colouring and ruling sets. In \cite{johnson1985symmetry} Johnson and Schneider present a discussion to address the meaning of symmetry and introduce the concept of similarity.  
\newline
% \newpage



\section{The Symmetry Breaking Problem}
\label{cap:2}

A symmetric system can be defined as a system in which the processes are in an equivalent relationship, this means that if processes run the same code, it is possible to permute the processes without changing the behaviour of the system. One example of this state would be a ring in which there is no unique identifier for every process \cite{boldi1996symmetry}.

If we consider a message passing system in which exists an initial state of symmetry between all processes, there is an admissible synchronous execution for which the processes will continue in the same initial state. In this case, it is necessary a mechanism to break the symmetry, otherwise, the system cannot escape from the initial state.
%as shown in \cite{angluin1980local} by Angluin.

Symmetry breaking is one of the most extensively studied problems in distributed computing. The fundamental symmetry breaking problems on graph include the maximal matching, vertex colouring, ruling sets and \textit{MIS}. The last one can be considered as the central problem because all the others can be reduced to it, as shown in  \cite{linial1992locality}.Two problems related to the maximal independent set are studied in the literature. Finding the \textit{MIS} of a general graph and listing all the \textit{MISs} in a given graph. In the next section, a formal definition of the maximal independent set is given. A  discussion on the sequential approach vs the distributed approach is presented with the theoretical analysis on time complexity and message complexity in the case of the distributed approach.     

\subsection{Maximal Independent Set}

\theoremstyle{definition}
\begin{definition}

Given an undirected graph $G = (V,E)$, a Maximal Independent Set \textbf{MIS} is a set of vertices $S \subseteq V$ if it satisfied the following properties:   

\begin{enumerate}
  \item the set MIS is an independent set meaning that no two vertices $v,u \in S$ are adjacent,
  \item the set S is maximal, with regard to independence, meaning that for each vertex $v \notin MIS$, there exists a neighbour $u$ of $v$ such that $u \in MIS$.
\end{enumerate}

\end{definition}

Figure \ref{fig:graph1} shows an example of undirected an graph $G$ with 8 vertices and 14 edges. There are two \textit{MIS} in this example, showing that it is possible to find more that one solution for the same instance of $G$. The green nodes are part of the \textit{MIS}, in the first \textit{MIS} the number of nodes is 2 and 4 in the second solution. Note that no common edges exist between green vertices and every vertex that it is not part of the \textit{MIS} has a neighbour in it, satisfying the conditions set out above.
 
\begin{figure}[ht]
\centering
\includegraphics[width=1 \linewidth, height=5cm]{mis-example.PNG} 
\caption{Example of two Maximal Independent Set of a general graph}
\label{fig:graph1}
\end{figure}



The algorithm \ref{algorithm:secuential-mis} describes the general sequential algorithm to find the maximal independent set of a general graph. The time complexity is $O(N)$ since in the worst case, the algorithm has to check every vertex. Another approach to improve this time is desirable. In the next section, two distributed algorithms to find the \textit{MIS} are presented.



\begin{algorithm}
 \caption{Sequential Maximal Independent Set}
 \label{algorithm:secuential-mis} 

\SetAlgoNoLine
\KwResult{MIS Maximal Independent Set}
\KwData{ $G(V,E)$ Graph}
    \While {$V$ is not empty}{
        Choose a vertex $v \in V$
            Add $v$ to the set MIS\;
            Remove from $V$ the vertex $v$ and all its neighbours\;
        }
    
 
\end{algorithm}
 
 
\subsection{Distributed Maximal Independent Set}

A logarithmic lower bound is in general acceptable to consider a problem to be optimally solved. The difficulty to find it in sequential algorithms motivated the proposal of distributed algorithms. Deterministic and randomised algorithms are the approaches to solve problems in distributed computing.  Applying randomisation techniques on algorithms is a powerful and efficient technique to solve problems that may take longer time in a deterministic algorithm.

In 1986, an efficient distributed algorithm was proposed independently by Luby \cite{luby1986simple} and Alon \textit{et al.} \cite{alon1986fast}. Both algorithms are randomised and the expected termination time is $O(log N)$ rounds. It is worth to mention that all algorithms exposed below run in the synchronous model.  

The algorithm\ref{algorithm:luby-mis} describes the original Luby's algorithm. Prove the correctness of this algorithm is very simple. If a vertex joins the \textit{MIS}, in a round $r$, no other neighbour join the set in $r$ or in any other $r\prime$. The algorithm at the end produces a \textit{MIS} because the vertices with the highest degree will decide to enter the \textit{MIS} in each round until all vertices become inactive.

\begin{definition}

An algorithm terminates with high probability (w.h.p.) within $O(t)$ time if it does so with probability at least $1-\frac{1}{n^c}$ for any choice of $c\geq 1$.

\end{definition}

%  Message complexity depends on the number of processes which are active in each phase and its denoted by $O(m)$ in \cite{luby1986simple}. With high probability $\frac{1-1}{n}$, the Luby's algorithm finishes  $4\log N$ round. 

% \theoremstyle{theorem}
\begin{theorem}

Algorithm \ref{algorithm:luby-mis} computes a maximal independent set for any graph  in $O(\log n)$  rounds with high probability.

\end{theorem}


\begin{algorithm}
 \caption{Luby's Algorithm, code for each process $p_i$ $i = 1$ to $N$}
 \label{algorithm:luby-mis} 

\SetAlgoNoLine
\KwResult{MIS Maximal Independent Set}
\KwData{ $G(V,E)$ Graph}
    \While {V is not empty}{
        Choose a random set of vertices $S ⊆ V$, by selecting each vertex $v$ independently with probability $1/(2d(v))$, where d is the degree of $v$\;
        For every edge $e \in E$, if both its endpoints are in the random set $S$, then remove from $S$ the endpoint whose degree is lower. Break ties arbitrarily, e.g. using a lexicographic order on the vertex names\;
        Add the set $S$ to $MIS$\;
        Remove from $V$ the set $S$ and all the neighbours of vertices in $S$\;
        }
\end{algorithm} 



The algorithm \ref{algorithm:main-mis} is another randomised distributed algorithm and was proposed by Yves \cite{yves2009optimal}. This algorithm is used for the simulations in this project and it is a variation of Luby's algorithm. The rounds can be split into 2 phases for simplicity. In each phase, each process chooses a random value, send it to its neighbours and wait to receive the value from all its neighbours. If the process has the maximum value, the join the \textit{MIS}. In the second phase, if the $p_i$ decided to join the $MIS$, then notified its neighbours that $p_i$ is part of the \textit{MIS}. If $p_j$ receives the last notification message from $p_i$, then $p_j$ decide not to join the \textit{MIS}. At the end of this phase, every process that made a decision about joining or not the \textit{MIS} become inactive for the next rounds.

\begin{algorithm}
 \caption{MIS Algorithm, code for each process $p_i$ from $i = 1$ to $N$}
 \label{algorithm:main-mis} 

\SetAlgoNoLine
\KwResult{MIS Maximal Independent Set}
\KwData{ $G(V,E)$ Graph}
    \While {V is not empty}{
        $p_i$ select a random number $r(v)$ between [0,1] and sends to its neighbours\;
        If $r(v) < r(w)$ for all neighbours $w \in N(v)$ of $p_i$, remove myself from $V$ and enter to the MIS \newline
        Inform my neighbours that I am a MIS member and terminate\;
        If $p_i$ heard that my neighbour $p_j$ is in the MIS, remove myself from the $V$ and terminate\;
        }
\end{algorithm}

% Lemma 7.14 (Edge Removal). In a single phase, we remove at least half of
% the edges in expectation.

The correctness is very intuitive and similar to the Luby's algorithm. In one phase, one process $p_i$ joins the \textit{MIS} only if it has the largest value among its neighbours. At the end of that phase, all the neighbourhood of $p_i$ become inactive, including $p_i$ and there is no neighbour of $p_i$ in the \textit{MIS}. This set is maximal because at least one vertex (with the global smallest value) will enter into the \textit{MIS} per round, hence there is a progress in each round. If at some round, a vertex has no neighbours, it automatically joins the set and become inactive. This sequence continues in the following rounds until every process becomes inactive.

% \theoremstyle{theorem}
\begin{theorem}

Algorithm \ref{algorithm:main-mis} computes a maximal independent set for any graph in $O(\log n)$ rounds with high probability.

\end{theorem}


 Until now, these two algorithms are faster than the best deterministic algorithms for general graphs. There have been some improvements for special cases, for instance, \cite{panconesi1996complexity} proposed a $O(\Delta + log^* N)$ algorithm for specific graphs, however, the original algorithms are still faster when the running time is expressed as a function of $N$ and for any type of graph.
 
 In a round when a vertex join the \textit{MIS} all the edges of $v$ and the edges of any neighbour of $v$ are removed from $E$. In general, this number is much grater that the degree of $v$. So there is a high probability that the topology decreases very fast. Indeed, this is the mechanism to demonstrate the lower bound of $O(\log N)$. In the section \ref{chap:6} (Evaluation of results), this supposition is tested experimentally. 
 

% . This algorithm operates in synchronous rounds. In line 2, every process select a random number to in order to break the symmetry with its neighbours, line 3 makes sure that if a vertex v join the MIS, no other neighbour of v join the MIS at the same time, this is true because the execution  occurs in rounds. the line 4 makes sure that any vertex that has a neighbour in the MIS, join the MIS at any point. 


\newpage
% \newpage

\section{Synchronous Distributed Systems}
\label{chap:3}

In this model, processes partition the execution of algorithms into rounds. In each round, a processor can send messages, receive messages and perform some local computation. Synchronous systems come with guarantees properties on the nature of the system. For instance, upper bound on message delivery, ordered messages delivery, globally synchronised clocks, lock step based execution among others. 

Although it is difficult to implement a model with these strong assumptions, it is favourable for designing algorithms. The main problem with asynchronous systems is that message delay is unbounded, there is no limit on how long a process should wait in order to determine that it received all messages from its neighbours. Once an algorithm is designed for the synchronous model, it can be simulated in a more realistic model like an asynchronous system \cite{attiya2004distributed}.
% The reason why synchronous algorithms are desirable is that there are usually simpler to design and superior in complexity.

A synchronizer is a general technique to simulate synchronous communication in asynchronous systems. Synchronizer were introduced by Awerbuch \cite{awerbuch1985complexity}. The author presented 3 different synchronizers and analysed the trade off among them. This technique allows the execution of distributed synchronous algorithm over an asynchronous system, for instance, an asynchronous message passing system.  

The graphic \ref{fig:simulation} shows the interactions among the layers of the simulation. The bottom layer is the asynchronous message passing system, at this layer, there is no guarantees on messages delay. A synchronizer works on the top of this layer and its goal is to provide the illusion of a synchronous system to the upper layer. In the example, the user of the synchronizer is the Maximal Independent Set algorithm. Any user of the synchronizer can use the interface \textit{Sync-}$send_i$ and \textit{Sync-}$recv_i$ of the Synchronizer and safely assume that the communication is synchronous.     

\begin{figure}[ht]
\centering
\includegraphics[width=1 \linewidth, height=8cm]{simulation.PNG} 
\caption{Diagram of the simulation of a synchronous distributed systems using synchronizers}
\label{fig:simulation}
\end{figure}



The following model was used in the original paper by Awerbuch \cite{awerbuch1985complexity} and was previously used in \cite{segall1983distributed,gallager1982distributed} among others. The asynchronous system is represented by an undirected graph $G = (V,E)$ where $V$ is the set of processes and $E$ is the set of bi-directional communication channels. The local model is also used for the synchronizer implementation. The messages delay is assumed to be finite in order to start a new round but there is no restriction on how long should be this delay. Another important assumption of the model is that the amount of information carried by messages is limited. 

Processes send messages after they receive a pulse or clock in the synchronous model. This pulse represents one time unit (round) in the synchronous system. The delay of the messages in one round is one unit of time of the global clock.

\begin{definition}
\label{def:safe}
A process $p_i$ is said to be safe in a round only after all its messages has been delivered at their destinations.
\end{definition}

The notion of safe process was introduced in \cite{awerbuch1985complexity}. If all neighbours of $p_i$ are safe for the round $r$, this mean that $p_i$ has received all messages for $r$. In this case, $p_i$ in ready to execute the $r + 1$ round. This property ensures that from the point of view of the user of the synchronizer, the network behaves as a synchronous communication system when it is really an asynchronous system. An easy solution to detect if a process is safe, it is to force each process to acknowledge every message that receives. 

The  overhead generated by a Synchronizer $S$ with the acknowledgement mechanism is the double the number of messages of the original algorithm $A$. To compute the total message and time complexity of a synchronous algorithm it is necessary to sum the complexity of $A$ and $S$. $T(S)$ and $M(S)$ denote the time and message complexity per round respectively. If the synchronizer requires an initialization phase (for instance, the Beta synchronizer) $T_{init}(S)$ and $M_{init}(S)$ express the message and time complexity for the initialization phase.  $M(A)$ and $T(A)$ are  the time and message complexity of the algorithm $A$. For algorithm implemented over a synchronizer, the total message complexity is then expressed in the equation \ref{ec:mess} and the time complexity in the equation \ref{ec:time}. 


\begin{equation}
\label{ec:mess}
 T_{tot} = T_{init}(S) + T(A)(1+T(S)) 
\end{equation}

\begin{equation}
\label{ec:time}
M_{tot} = M_{init}(S) + M(A) + T(A)M(S) 
\end{equation}


The two synchronizers presented in the next section are denoted Synchronizer $\alpha$ and Synchronizer $\beta$ \cite{awerbuch1985complexity}, which are a generalisation of the implementation proposed by Gallager in \cite{gallager1982distributed}. These two synchronizers present a trade off between messages and time complexity. Synchronizer $\alpha$ is efficient in time but produce a significant overhead in messages, while Synchronizer $\beta$ has a better performance in communication but it is worse in time complexity.



\subsection{Alpha Synchronizer}

One design challenge with synchronizers is the unbounded messages delay on real time. In consequence, a process cannot detect when it is safe just waiting until receiving all messages. The Alpha synchronizer use the acknowledgement mechanism to solve this problem. 

In the Synchronizer $\alpha$, a process $p_i$ send an acknowledgement for each message that receives. After $p_i$ has received all acknowledgement messages from its neighbour, $p_i$ knows that it is safe with respect to the round $r$. When $p_i$ detect that it's safe, then send a message \textbf{<safe,round>} to all its neighbours. Only after $p_i$ learns that all its neighbours are safe can start a new round.

The code for Synchronizer $\alpha$ is described in the algorithm \ref{algorithm:alpha}. The pseudo--code was extracted from \cite{attiya2004distributed}.  



\begin{algorithm}
 \caption{Alpha Synchronizer, code for $p_i$ from $i = 1$ to $N$}
 \label{algorithm:alpha} 

\SetAlgoNoLine

Initially \textit{round} = 0 and \newline
\textit{buffer[r], safe[r]} and \textit{ack-missing[r]} are empty for all $r \geq 1$ \newline

\textbf{When} \textit{Synch-}$send_i$ (S) occurs:\newline
$round = round + 1$ \newline
\textit{ack-missing[round]} = {$j:p_j$ is a recipient of a message in S} \newline
enable \textit{Asynch-}$send_i(<m,round>)$  to $p_j$, for each $m \in S$ with recipient $p_j$ \newline

\textbf{When} \textit{Asynch-}$recv_i(<ack,r>)$ from $p_j$ occurs: \newline
add $(m,j)$ to \textit{buffer[r]} \newline
enable $Asynch-send_i(<ack,r>)$ to $p_j$ \newline

\textbf{When} \textit{Asynch-}$recv_i(<ack,r>)$ from $p_j$ occurs: \newline
remove \textit{j} from \textit{ack-missing[r]} \newline
\If{ack-missing[r] = 0}{ 
enable \textit{Asynch-}$send_i(<safe,r>)$ to all neighbours \newline
}

\textbf{When} \textit{Asynch-}$recv_i(<safe,r>)$ from $p_j$ occurs: \newline
add \textit{j} to \textit{safe[r]} \newline
\If{safe[r] includes all neighbours}{
  enable \textit{Synch-}$recv_i(buffer[r])$ \newline
}

\end{algorithm}


In each round $p_i$ send extra messages to each neighbour, the equation \ref{ec:message-alpha} describe the message complexity of Synchronizer $\alpha$ per round.  For this synchronizer, $p_i$ needs one additional time unit in order to detect that it is safe, therefore the time complexity is constant, see equation \ref{ec:time-alpha}.  


\begin{equation}
\label{ec:message-alpha}
 C(\alpha) = O(E) = O(V^2) 
\end{equation}

\begin{equation}
\label{ec:time-alpha}
 T(\alpha) = O(1) 
\end{equation}


\subsection{Beta Synchronizer}

The Synchronizer $\beta$ is similar to the previous synchronizer. The difference with the Synchronizer $\alpha$ is the safe detection mechanism. This synchronizer needs an initialisation phase in which a rooted spanning tree is constructed over the network topology. A leader $S$ has to be chosen and then the spanning tree is constructed from the root. The acknowledgement mechanism is the same but instead of informs all its neighbours, $p_i$ just send the safe message to the parent in the spanning tree, this process is call convercast. A process $p_i$ is safe when has received the acknowledgement for each message sent in the actual round plus an additional \textbf{<safe,round>} from all its children in the spanning tree. Initially only processes that are on the bottom of the spanning tree (the leaves) send the message \textbf{<safe,round>} and progressively the parents of the leaves send this message to their parents once they are safe. When the root is safe, then broadcast \textbf{OK} on the spanning tree and every process is allowed to pursue with the next round. This blow up the time overhead since the mechanism acts as a global synchronizer with a central leader, which is the process acting as the root of the spanning tree. However, $p_i$ only send one safe message instead of sending to all its neighbourhood improving the message overhead of Alpha. 


\begin{algorithm}
 \caption{Beta Synchronizer, code for $p_i$ from $i = 1$ to $N$}
 \label{algorithm:beta} 

\SetAlgoNoLine

Compute a rooted spanning tree
Initially \textit{round} = 0 and \newline
\textit{buffer[r], safe[r]} and \textit{ack-missing[r]} are empty for all $r \geq 1$ \newline

\textbf{When} \textit{Synch-}$send_i$ (S) occurs:\newline
$round = round + 1$ \newline
\textit{ack-missing[round]} = {$j:p_j$ is a recipient of a message in S} \newline
enable \textit{Asynch-}$send_i(<m,round>)$  to $p_j$, for each $m \in S$ with recipient $p_j$ \newline

\textbf{When} \textit{Asynch-}$recv_i(<ack,r>)$ from $p_j$ occurs: \newline
add $(m,j)$ to \textit{buffer[r]} \newline
enable-$Asynch-send_i(<ack,r>)$ to $p_j$ \newline

\textbf{When} \textit{Asynch-}$recv_i(<ack,r>)$ from $p_j$ occurs: \newline
remove \textit{j} from \textit{ack-missing[r]} \newline
\If{ack-missing[r] = 0}{ 
enable \textit{Asynch-}$send_i(<safe,r>)$ to my parent in the spanning tree \newline
}
\If{$p_i$ is the root}{ 
enable \textit{Asynch-}$send_i(<go,r>)$ all my childrens in the spanning tree \newline
}

\textbf{When} \textit{Asynch-}$recv_i(<go,r>)$ from $p_j$ occurs: \newline
  enable \textit{Synch-}$recv_i(buffer[r])$ \newline
  enable \textit{Asynch-}$send_i(<go,r>)$ all my childrens in the spanning tree \newline

\end{algorithm}

The diameter of the spanning tree is $N - 1$, so the convercast and broadcast take at most $2N - 2$ time units combined. Time and message  complexity per synchronous round are expressed in the equations \ref{ec:message-beta} and \ref{ec:time-beta} respectively. The initialisation phase only needs to be done one time for each topology, for this reason it is more interesting the overhead $T(\beta)$ and $M(\beta)$. The time and message complexity of the initialization phase are $T_{init}(\beta) = O(V)$ and $M_{init}(\beta) = O(M + N \log N)$  respectively.

\begin{equation}
\label{ec:message-beta}
 C(\beta) = O(V)
 \end{equation}

\begin{equation}
\label{ec:time-beta}
 T(\beta) = O(V) 
\end{equation}

\subsection{Discussion about Synchronisation techniques}

One synchronizer is efficient in terms of time and the other in term of messages. The same author proposes \cite{awerbuch1985complexity} another synchronizer that tries to get a low overhead in both, the Synchronizer $\gamma$. Essentially, the idea is to generate a spanning forest of the graph and run Beta within each tree and Alpha between trees. If there are not too many adjacent trees, the messages overhead is similar to the Beta Synchronizer. The time overhead is proportional to the depth of the trees. In some special cases  (depending on the topology of the graph, for instance, a ring of k-cliques \cite{lynch1996distributed}), the Gamma Synchronizer can present a similar cost to the original synchronous algorithm. However, it is also possible the need of tune the spanning forest for some types of graphs. Besides, the Gama Synchronizer is more complex in terms of implementation and initialization phase.

All these synchronizers require that the entire network participate in the synchronization process, even if some processes have no message for some round. The problem is when $p_i$ do not send any messages in one round, then any process $p_j$ that is a neighbour of $p_i$ cannot deduce this situation because the delays in asynchronous networks are unbounded, in consequence, the use of timers are not useful. One solution is  to send dummy message in order to keep the synchronization. As a result, the overhead is always linear on the number of processes. Awerbuch and Peleg proposed in \cite{awerbuch1990network} a poly-logarithmic overhead synchronizer based on involving only the relevant portion of the topology in the synchronization process. However, it is possible that the synchronizer requires high space complexity.  Another approach for dynamic networks can rely on compute for each process the actives neighbours in every round and apply a simple synchronizer, Alpha for instance. This approach study in \cite{AspnesW2007} requires that each $p_i$ computes all neighbours before the execution of each round.

For the simulations in this project, besides the Beta and Alpha Synchronizers, another global synchronization mechanism is used. A master process $M$ is required to control the synchronization between every active process on the network.  Each process must inform $M$ that has finished the computation for the round $r$ and then $M$ is in charge of notifying active processes that are allow to start the $r + 1$ round. This mechanism generates a lot of computation in the master process, however, no unnecessary message is sent by inactive process. 

In the case of the Distributed Maximal Independent Set problem, especially with the algorithms explained in this chapter, this behaviour is desirable because many processes become inactive very quickly. These three techniques are used to simulate a synchronous communication for the \textit{MIS} algorithm. The analysis of the trade-off among them is presented in section \ref{chap:6}.





% \newpage

\section{Simulator}
\label{chap:4}



\subsection{Code Structure}

% \newpage

\section{System Evaluation}

\subsection{Evaluation Criteria}

In this section, the results of the simulation are presented. The randomised parallel algorithm proposed in \cite{yves2009optimal} is tested with three synchronization techniques using the simulator described previously. The theoretical bound for this algorithm is known to be $O(\log N)$. The synchronizers time and message complexity were discussed in section \ref{chap:3}. 

HERE INSERT DEFINITION OF TIME AND MESSAGES COMPLEXITY. CHECK IF THERE IS ALREADY IN THE INTRODUCTION.

 The aim of the simulation is to evaluate the time and message complexity of the algorithm. Additionally, show how the synchronization techniques generate overhead over the synchronous algorithm and present a discussion about the trade off between the techniques based on experimental results.


\subsection{Network Models}
\label{sec:topology}


The network topologies need to be generated before testing the \textit{MIS} algorithm. Topologies represent the distributed system in which the algorithm is going to be tested. Processors are mapped to vertices and the communication channels to edges.  The random graph model used to generate the networks topologies is the Stochastic Block Model (SBM), proposed in \cite{holland1983stochastic}.

Before entering in details on the SBM, a brief explanation about the Erd\~os--R\'enyi model is presented. This model is used to generate random graphs and it was first proposed in \cite{erdds1959random}. There are two variants of the model, in the $G(n, p)$ model a graph is constructed by connecting $n$ vertices randomly. An edge is included in the graph between two vertices $i,j$ with independent probability $p$. On \cite{erdos1960evolution} Erd\~os and R\'enyi presented some important properties about random graph. These properties about connectivity are used to generate the topologies for the simulations of this project and are cited below.


\begin{enumerate}
\item If $p<{\tfrac {(1-\epsilon )\ln n}{n}}$, then a graph $G(n, p)$ it is very likely to be disconnected.
\item If $p>{\tfrac {(1+\epsilon )\ln n}{n}}$ , then a graph $G(n, p)$ it is very likely to be connected.
\end{enumerate}

The most important point for this project is that ${\tfrac {\ln n}{n}}$ is a threshold of $p$ for generate graphs $G(n, p)$ that will almost surely be connected. 

In the SBM, the networks are characterised by blocks structures. Blocks define partitions of the networks, this means that vertices are associated in different subgroups and the distribution of the edges between vertices depend on the block in which a vertex is member. Assigning different probabilities to blocks, it is possible to obtain graphs that are denser in some regions. The probability for each group is defined in a probability matrix. In the next section, a formal definition of the \textit{SBM} is given with some examples of graphs generated by this model.

\subsection{Stochastic Block Model}

The SBM can be considered a probabilistic or generative model in which a probability is assigned to each pair of vertices $i,j$ in the network. This model define a probability distribution over networks $Pr(G | \theta)$, where $\theta$ are the parameters for the edges probabilities. For a given $\theta$, it is possible to generate a network instance G from the distribution by flipping a coin between every possible pair of vertices of the $G$. 

The Stochastic Block Model is defined by: 
\begin{enumerate}

    \item $k$: a scalar representing the number of blocks groups, clusters or modules in the network.
    \item $\overrightarrow{z}$: a vector of n element where $z(l)$ gives the block index of the vertex $l$.
    \item $M$ is a $k * k$ stochastic block matrix, where $M_{ij}$ gives the probability that a vertex of block type $i$ is connected to a vertex of block type $j$.
\end{enumerate}


The edges generated by flipping the coin between each pair of vertices are independent but not identically distributed. The large number of parameters allows the SBM  to produce very different structures. If $M_{ij} = p$ for each pair of blocks, then the SBM is reduced to the Erd\~os--R\'enyi model of random graph $G(n,p)$. The figure \ref{fig:erdos} show an example of stochastic block matrix in which every probability is the same for $k = 3$ blocks. Note that this is a square matrix and $k$ should be defined before $z$ and $M$. It can be seen that the probability distribution is the same for each block, as a consequence, the graph looks like one component even if the vertex belongs to different blocks types.

 In the figure \ref{fig:sbm}, another example of random graph is shown, however, in this example the probability distribution differ between blocks and it is more easy to visualize the different groups in the picture. The entrance $M_{i,j}$ where $i = j$ represent the probability of one edge between two vertices in the same group and the entrance $M_{i,j}$ where $i \neq j$ represent the probability of one edge between two vertices in the different groups. 

When $M_{i,i}$ is greater than $M_{i,j}$ for $i \neq j$ the vertices tend to connect other vertices that are in the same group. For this configuration the values of the diagonal are larger than values off it, in this case, it is the communities are assortative and the edges are more common within the blocks. On the contrary, on disassortative communities, edges between vertex of different communities are more common $M_{ii} <  M_{ij}$ for $i \neq j$.


Many other grouping criteria and variation of \textit{SBM} exist \cite{carrington2005models,holland1983stochastic,airoldi2008mixed}. Particularly, for this project, the assortative grouping is used for generating topologies from $N = 2^{6}$ to $2^{15}$ with power increments of 2 in the size of the network. The ${\tfrac {\ln n}{n}}$ threshold is used to set the probability of the diagonal in the matrix of probabilities. The values used in the parameters are similar to the used in \cite{kothapalli2013analysis}. 




\begin{figure}[ht]
\centering
\includegraphics[width=1 \linewidth, height=5cm]{Erdos-Renyi.PNG} 
\caption{Example of SBM with equal probabilities}
\label{fig:erdos}
\end{figure}

\begin{figure}[ht]
\centering
\includegraphics[width=1 \linewidth, height=5cm]{sbm.PNG} 
\caption{Example of SBM with assortative communities}
\label{fig:sbm}
\end{figure}

\subsection{Description of Experiments}

As mention in section \ref{chap:4}, the simulator was implemented in Elixir 1.6. The topology generator used to create network topology is an open source software written in Python under MIT license. Some modifications were done to the generator like generate the edges only for the triangular upper matrix, modified the number of clusters, customised the file format for topologies and some other minors modifications. The original code can be found here \url{https://github.com/tcoyze/stochastic-blockmodel} and the generator with the modification is inside the project repository.


The tests were run in a computer with processor Intel(R) Core(TM) i5--4210U of 1.7 and 2.4 GHz and RAM memory of 8 GB with additional 8 GB of SWAP. The operating system was a Linux 64 bit Debian base.

Regarding with network topologies, ten different topologies were generated with the Stochastic Block Model $G = (n,p)$ starting with $n=64,256,512,...,32768$. The number of blocks $k$ is set to $\log n$. The value used for the probability $p$ on the diagonal of the block matrix was $7{\tfrac {\ln n}{n}}$  and for values out of the diagonal $p\prime = {\tfrac {10}{n}}$. The value for $p$ was chosen to make sure that the generated graph is connected. Note that the algorithm for \textit{MIS} does not require this condition because it still reaches the solution even if the graph is disconnected. However, the Beta Synchronizer construct a rooted spanning tree over the topology in order to send \textbf{safe} and \textbf{go} messages to the root and from it. Because of this, if the topology is not connected the rooted spanning tree can not be constructed.


For each topology, the results presented are the average 10 executions of the simulation. For the \textit{MIS } algorithm it is desirable to measure the average because the randomised nature of the algorithm and the unpredictable behaviour of the asynchronous message passing at the bottom. In consequence, for a given topology it is possible to obtain different results. The results observed in the simulation show that the number of rounds is similar in different execution however in can be some important differences in the number of messages. These results are presented in the next section.

The steps to perform the experiments are enumerate below:

\begin{enumerate}
\item Implementation of the synchronizers Alpha and Beta.
\item Implementation of the Maximal Independent Set algorithm using the simulation developed in step 1.
\item Implementation of the Maximal Independent Set Algorithm using a global synchronization technique.
\item Create the topology with the Stochastic Block Model according to the description in the section \ref{sec:topology}.
\item Run N times the synchronous algorithm for the \textit{MIS} and saves the results for each synchronizer technique (Alpha, Beta and Global).
\item Compute the average and present the results.
\end{enumerate}





% \newpage

\section{Result Evaluation}
\label{chap:6}
Regarding the network topologies, ten different topologies were generated with the Stochastic Block Model for $n = 64, 256, 512,..., 32768$. The number of blocks $k$ was set to $\log n$. The values were set to $p = 7{\tfrac {\ln n}{n}}$ for the diagonal of the probability matrix and  $p\prime = {\tfrac {10}{n}}$ for values out of the diagonal. The values for the parameters are similar to the values used in \cite{kothapalli2013analysis}.

First, the result of the evaluation of the \textit{MIS} algorithm is presented. Then, the analysis of the synchronisation techniques and finally the conclusion of the evaluation.

% The results presented are the average of 10 executions of the simulation for each topology. For the \textit{MIS} algorithm it is important to measure the average of the execution because the algorithm is randomised and the unpredictable behaviour of the asynchronous message passing at the bottom. In consequence, for a given topology it is possible to obtain different results. The results observed in the simulation show that the number of rounds is similar in different execution however in can be some important differences in the number of messages. These results are presented in the next section.


\subsection{Evaluation of MIS algorithm}

The figure \ref{fig:rounds_execution} shows the average number of round that takes on each topology to finish the \textit{MIS} algorithm. As seen in section \ref{cap:2}, the expected termination time of the algorithm is $O(\log N)$ rounds. The plot is compared with a logarithmic plot showing similarity in shape of curves.The figure \ref{fig:rounds-erdos} shows the average number of round per topology using network topologies generated by the Erd\~os--R\'enyi model with  $p = 5{\tfrac {\ln n}{n}}$. The shape of the curve also is similar to the logarithmic curve shape for graphs with a constant probability distribution of edges.


\begin{figure}[ht]
\centering
\includegraphics[width=1 \linewidth, height=6cm]{number_rounds.png} 
\caption{Rounds per execution vs log(x) curve}
\label{fig:rounds_execution}
\end{figure}



\begin{figure}[ht]
\centering
\includegraphics[width=1 \linewidth, height=8cm]{execution-rounds-erdos.PNG} 
\caption{Rounds of MIS algorithm using Erdos-Renyi model for network topologies}
\label{fig:rounds-erdos}
\end{figure}


The distributed algorithm should make a progression in each round on the number of processes that finish the local algorithm and become inactive according to the proof of termination in \cite{yves2009optimal}. This progression for a network of 32768 processes can be seen in the figure \ref{fig:progression} . The blue line represents the number of processes that join the \textit{MIS}  per round and the red line represents the number the neighbours of some process that took part of the \textbf{MIS}. The figure \ref{fig:actives}, shows the total number of processes that are active in each round for the same execution. By the end of round 14, the network has no active processes.  

\begin{figure}[ht]
\centering
\includegraphics[width=1 \linewidth, height=8cm]{progress.PNG} 
\caption{Numbers of processes that finish the execution per round (MIS and not MIS)}
\label{fig:progression}
\end{figure}

\begin{figure}[ht]
\centering
\includegraphics[width=1 \linewidth, height=8cm]{actives_round.PNG} 
\caption{Total numbers of processes that finish the execution per round}
\label{fig:actives}
\end{figure}


\subsection{Evaluation of Synchronizers}

To evaluate the messages send by the algorithm first, we need to make a distinction between messages sent by the synchronous algorithm and the additional messages send by the synchronizers. In the case of Alpha and Beta, it is straightforward to count this two types of messages. Every time a process use the \textbf{sync-send} function, it is count as one message sent by the algorithm. For every synchronous message, the synchronizer sends some asynchronous messages which are multiple of its neighbour size and also depend on the mechanism used to detect when a process is safe. For the Global synchronizer, the counting is a little more complicated. The messages for acknowledgement, to update the topology and the notifications to the master process count as synchronisation overhead.


The figure \ref{fig:total_msg} illustrate the messages send by each synchronizer for the network of 32768 processes. The orange bar represents the synchronisation overhead and the blue bar the messages sent by the synchronous algorithm. It is clear that the global synchronizer is the one with the best performance. In this case, the synchronisation adds around $20 \%$ more messages. For the other two, the difference is overwhelming. The reason is that the Global synchronizer is optimal concerning the messages send by actives processes. In each round, once a process becomes inactive, no additional messages are sent by this process. On the contrary, Alpha and Beta require the entire topology to participate in every round until the distributed algorithm end.  The figure \ref{fig:total_msg-round} shows how the number of messages decreases in each round for the global synchronizer, however, for Alpha and Beta remain constant. Finally, the figure \ref{fig:total_msg-size} shows the total number of messages send by the algorithm for different networks size. The messages complexity is bounded by the number of edges of the topology and a constant factor. The three functions show an approximate the shape of a line in agreement with the theoretical analysis.





\begin{figure}[ht]
\centering
\includegraphics[width=1 \linewidth, height=7cm]{messages-synchroniser1.PNG} 
\caption{Messages send by the algorithm and Synchronizer}
\label{fig:total_msg}
\end{figure}

\begin{figure}[ht]
\centering
\includegraphics[width=0.9 \linewidth, height=6cm]{messages-rounds.PNG} 
\caption{Messages send by the algorithm in different rounds}
\label{fig:total_msg-round}
\end{figure}

\begin{figure}[ht]
\centering
\includegraphics[width=0.8 \linewidth, height=6cm]{messages-network.PNG} 
\caption{Total Messages vs Network size}
\label{fig:total_msg-size}
\end{figure}


\subsection{Conclusions}

In this dissertation, the problem of finding the Maximal Independent Set in general graphs was studied. Distributed algorithms provide better time complexity than sequential algorithm.  A distributed synchronous algorithm was implemented in order to analyse time and message complexity for the \textit{MIS} problem. Synchronous algorithms are easy to design and have lower complexity than asynchronous ones.

\newpage

% \newpage

\section{Self Assessment}
\label{chap:7}


In this section, I will present a personal evaluation of this final dissertation.  The assessment includes the progression of the project, acquired skills and a description of professional issues, in particular about correct citation and code acknowledgement.

% I started with this project at the end of May. I began to read about the symmetry breaking problem in distributed computing and implemented 2 leader election algorithms (studied in CS5860) for ring topologies. On the first week of June, I decided to work on the Maximal Independent Set problem. After read several papers and online materials, I started implementation of the algorithm proposed in \cite{yves2009optimal} with a global synchronisation technique. By the end of June, I began with the synchronizers Alpha and Beta. After I finished the synchronizers, I focused on the report and in minors changes to the implementation.

During this project, I was able to consolidate my knowledge of programming distributed systems. In the course CS5860, I had the opportunity to learn the Actor model in Elixir and implement programs on the distributed model assuming asynchronous communication. Besides, in the Wireless Sensor Networks course (CS5870), I implemented two gossip algorithm over a stack of Arduino and XBee. I learnt the importance of theoretical analysis for distributed algorithms. The proof of correctness and termination ensure that an algorithm is correct. In practice, debug an algorithm for large topologies is almost impossible, but one can rely on the proofs to make sure the correctness of the implementation. These experience helped me to understand the model of distributed programming and prepared me to face this dissertation. After completing this project, I feel more confident in my ability to implement and design distributed applications and feel that I gained a lot of practical experience and theoretical knowledge.     

% I learnt the importance of theoretical analysis for distributed algorithms. The proof of correctness and termination ensure that an algorithm is correct. In practice is also impossible  to debug an algorithm for big topologies but one can relay on the proofs to make sure the correctness of the implementation. 
% theoretical knowledge gain during the course help me to solve the problem. 

I used open-source code to generate the network topologies for the testbed. First, I used the open source library (GPLv3) \textbf{algs4} written in Java, which is part of the book \textbf{Algorithms $4^th$ Edition} of Sedgewick and Wayne. This library has a section for generating random graphs which include the model  Erd\~os--R\'enyi. Finally, I used another open-source implementation from a public GitHub repository under MIT license to produce the network topologies with the Stochastic Block model. It is essential to mention and acknowledge in the report any implementation that was not made by me. The correct citation was done in the section \ref{chap:5}. I had to spend some time reading about licensing to understand until what point I could reuse the software. The correct citation is also another important topic regarding ethical issues. I read the rules of the University on plagiarism and I used online tools to double check if there was any part of the report which was not properly cited. 

My contribution to the community is a new simulator of synchronous communication written in Elixir in which anyone can test synchronous algorithm without worrying about the underline asynchronous communication.  This simulator is in the public domain in GitHub \href{https://github.com/mtileria/SymmetryBreaking}. I decide to put this implementation under GPLv3 license, to make sure that this code will remain free to anyone.

% Plagiarism - correct citation, using code with acknowledgement.
% \newpage


\bibliographystyle{plain}
\bibliography{bibliography}
\end{document}
